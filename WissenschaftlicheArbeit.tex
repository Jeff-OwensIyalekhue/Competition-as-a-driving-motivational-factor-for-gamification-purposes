%%%%%%%%%%%%%%%%%%%%%%%%%%%%%%%%%%%%%%%%%%%%%%%%%%%%%%%%%%%%%%%%%%%%%%%%%%%%%%%%
% TUM-Vorlage: Wissenschaftliche Arbeit
%%%%%%%%%%%%%%%%%%%%%%%%%%%%%%%%%%%%%%%%%%%%%%%%%%%%%%%%%%%%%%%%%%%%%%%%%%%%%%%%
%
% Rechteinhaber:
%     Technische Universität München
%     https://www.tum.de
% 
% Gestaltung:
%     ediundsepp Gestaltungsgesellschaft, München
%     http://www.ediundsepp.de
% 
% Technische Umsetzung:
%     eWorks GmbH, Frankfurt am Main
%     http://www.eworks.de
%
%%%%%%%%%%%%%%%%%%%%%%%%%%%%%%%%%%%%%%%%%%%%%%%%%%%%%%%%%%%%%%%%%%%%%%%%%%%%%%%%

%%%%%%%%%%%%%%%%%%%%%%%%%%%%%%%%%%%%%%%%%%%%%%%%%%%%%%%%%%%%%%%%%%%%%%%%%%%%%%%%
\input{./Ressourcen/Praeambel.tex} % !!! NICHT ENTFERNEN !!!
%%%%%%%%%%%%%%%%%%%%%%%%%%%%%%%%%%%%%%%%%%%%%%%%%%%%%%%%%%%%%%%%%%%%%%%%%%%%%%%%

\renewcommand{\Thema}{%
    Competition as a driving motivational factor for gamification purposes}

%%%%%%%%%%%%%%%%%%%%%%%%%%%%%%%%%%%%%%%%%%%%%%%%%%%%%%%%%%%%%%%%%%%%%%%%%%%%%%%%
\input{./Ressourcen/Anfang.tex} % !!! NICHT ENTFERNEN !!!
%%%%%%%%%%%%%%%%%%%%%%%%%%%%%%%%%%%%%%%%%%%%%%%%%%%%%%%%%%%%%%%%%%%%%%%%%%%%%%%%

\begin{document}

\title{\Thema}
\author{Jeff-Owens Iyalekhue}
\date{Datum}


\tableofcontents % Inhaltsverzeichnis

\chapter{Abstract}
\chapter{Introduction}

\section{Motivation}
\section{Definition}
\subsection{Gamification}
Gamification can be described as adding features of games to an other scenario. This is process has no further classifications and the purpose of this doing is to improve the modified setting. It is proclaimed adding characteristics of games to a training environment, is rising the interest of the users and learning increases.
\subsection{Competition}
Competition has to be divided into intrinsic and extrinsic , so you can compete against your own performance or you stand in competition with someone or something else.
\chapter{Approach}
To observe the influence of competition in the performances, we need a environment where subjects have to perform task in  scenarios facing different aspects of competition.
\section{Implementation}
\subsection{First Iteration}
implementing base game\newline
adding network features\newline
adding visual effects\newline
'halb coop' mode \newline
versus mode\newline
party mode \newline
collab mode\newline
further improvement of visual (on going)\newline
usability improvements\newline
log files\newline
\subsection{Pre-Study}
first data impression (no learning in later turns?)
bug report
\subsection{Second Iteration}
bug fixes\newline
removal of animations\newline
scripted player\newline
\chapter{User Study}
\chapter{Conclusion}

\end{document}
